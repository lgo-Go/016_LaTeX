\documentclass{article}
\usepackage[utf8]{inputenc}
\usepackage[russian]{babel}
\usepackage{amsmath}
\usepackage{hyperref}
\usepackage{amssymb}
\numberwithin{equation}{section}

\begin{document}
    
    \begin{center}
        {\Large МОСКОВСКИЙ ГОСУДАРСТВЕННЫЙ УНИВЕРСИТЕТ\\
        имени М.В. ЛОМОНОСОВА\\
        МЕХАНИКО-МАТЕМАТИЧЕСКИЙ ФАКУЛЬТЕТ}
    \end{center} 
    
    \vspace{4cm}
    \begin{center}
        {\LARGE {\bf A. В. Чашкин}}
    \end{center}
    
    \begin{center}
        {\LARGE \bf{ЛЕКЦИИ\\
        ПО ДИСКРЕТНОЙ МАТЕМАТИКЕ}}
    \end{center}
    
    \begin{center}
         {\bf Учебное пособие}
    \end{center}
    \newpage
    
    \tableofcontents
    \newpage
    
    \section{Число неприводимых многочленов}
    
    Применим метод производящих функций для нахождения числа неприводимых многочленов над полем $\mathbb{Z}_p$.
    Число неприводимых многочленов степени~$n$, укоторых коэффициент при старшей степени равен единице, обозначим через $P(n)$.
    
    \newtheorem{Lem}{Лемма}
    \begin{Lem}
       Для последовательности $P(n)$ справедливо рекуррентное равенство
       \label{lem.1}
    \end{Lem}
    
    \begin{equation}
        p^n = \sum_{m|n} m P(m).
        \label{eq.1}
    \end{equation}
    
    {\LargeДоказательство}.
    Пусть $p_{1m}, p_{p2m}, \cdots ,p_{P(m)m}$ --- все неприводимые многочлены степени $m$.
    Нетрудно видеть, что, раскрывая скобки в произведении
    
    \begin{equation}
        \prod_{m=1}^\infty \prod_{k=1}^{P(m)} \Bigl(1+p_{km}+(p_{km})^2+\cdots+(p_{km})^l+\dots \Bigl),
        \label{eq.2}
    \end{equation}
    
    \noindent получим сумму $\sum$ всевозможных произведений неприводимых многочленов, причем каждое произведение встретится в этой сумме ровно один раз.
    Так как каждый многочлен единственным образом раскладывается в произведение неприводимых многочленов,
    то в $\sum$ будет содержаться ровно $p^n$ произведений степени $n$. 
    Каждому неприводимому многочлену степени $m$ поставим в соответствие одночлен $x^m$, а произведению~\eqref{eq.2} --- произведение
    
    \begin{equation}
        \prod_{m=1}^\infty \prod_{k=1}^{P(m)} \Bigl(1+x^m+(x^m)^2+\cdots+(x^km)^l+\dots \Bigl)
        =\prod_{m=1}^\infty \left(\frac{1}{1-x^n}\right)^{P(m)}.
        \label{eq.3}
    \end{equation}
    
    \noindent Так как существует ровно $p^n$ многочленов степени $n$, укоторых коэффициент при $x^n^$ равен единице,
    то легко видеть, что в ряду, получившемся после раскрытия скобок в~\eqref{eq.3}, коэффициент при $x^n^$ будет равен $p^n$. Следовательно,
    
    \begin{equation}
    \frac1{1-px} = \prod_{m=1}^\infty \left(\frac{1}{1-x^n}\right)^{P(m)}.
        \label{eq.4}
    \end{equation}
    
    \noindent Логарифмируя правую и левую части~\eqref{eq.4}, получим новое равенство
    
    \begin{equation}
        \ln \frac1{1-px} = \sum_{m=1}^\infty P(m) \ln \frac1{1-x^n}.
        \notag
    \end{equation}
    
    \newpage
    \noindent Теперь, применяя формулу $\ln \frac1{1-x} = \sum_{n=1}^\infty \frac1n x^n$, разложим в ряд правую и левую части последнего равенства:

    \[
        \sum_{n=1}^\infty \frac1n p^n x^n = \sum_{m=1}^\infty \sum_{k=1}^\infty \frac1k P(m) x^{km} = \sum_{n=1}^\infty \left(\sum_{km=n} \frac1k P(m) \right) x^n.
    \]
    
    \noindent Приравнивая в получившемся равенстве коэффициенты при $n$-й степени $x$, находим
    
    \[
        \frac1n p^n = \sum_{km=n} \frac1k P(m) = \sum_{m\,|\,n}  \frac mn P(m).
    \]
    
    \noindent Лемма доказана.
    
    \vspace{5mm} \noindent Для того, чтобы из равенства~\eqref{eq.1} в явном виде выразить функцию $P(n)$ воспользуемся формулой обращения Мебиуса,
    которую докажем далее в лемме~\eqref{lem.3}. Сначала определим функцию Мебиуса
    
    \[
        \mu(m) = 
        \begin{cases}
            1, & \text{если $n=1$;} \\
            (-1)^k, & \text{если $n$ --- произведение $k$ простых чисел;}\\
            0, & \text{если $n$ делится на квадрат простого числа,}
        \end{cases}
    \]
    
    \noindent и покажем, что имеет место следующее утверждение.
    
    \newtheorem{Lem}{Лемма}
       \begin{Lem}
           Справедливо равенство
       \label{lem.2}
    \end{Lem}
    
     \begin{equation}
        \sum_{m|n} = 
        \begin{cases}
            1, & \text{если $n=1$;} \\
            0, & \text{если $n>0$.}
        \end{cases}
        \label{eq.5}
    \end{equation}
    
    {\LargeДоказательство}.
    Если $n=1$, то единица является единственным делителем, и, следовательно, $\mu(1) = 1$.
    При $n>1$ представим $n$ в виде произведения простых чисел:$n=p^{q_1}_1 \cdots p^{q_r}_r$. 
    Легко видеть, что в сумме \eqref{eq.5} нужно учитывать только делители без кратных множителей. 
    Поэтому
    
    \[
        \sum_{m|n} \mu(m) = \sum_{k=0}^r \,\, \sum_{1\le i_1<\cdots<i_k\le r} \mu(p_{i_1} \cdots p_{i_k}) = \sum_k^r = \sum \dbinom rk (-1)^k = 0.
    \]
    
    \noindent Лемма доказана.
    
    \newpage
    
    \newtheorem{Lem}{Лемма}
    \begin{Lem}
        Функци $f(n)$ и $h(n)$, определенные на множестве целых положительных чисел, удовлетворяют равенству
        
        \begin{equation}
            f(n) = \sum_{m|n}h(m) \textit{\hspace{8mm} при всех } n\in \mathbb{N}
        \label{eq.6}
        \end{equation}
        
        тогда и только тогда, когда
        
        \begin{equation}
            f(n) = \sum_{m|n} \mu (\frac nm) f(m) \textit{\hspace{8mm} при всех } n\in \mathbb{N}
        \label{eq.7}
        \end{equation}
        
    \label{lem.3}
    \end{Lem}
    
    {\LargeДоказательство}.
    Покажем, что из~\eqref{eq.6} следует\eqref{eq.7}. 
    Для этого прежде всего заметим, что
    
    \[
       \sum_{m|n} \mu (\frac nm) f(m) = \sum_{m|n} \mu(m) f(\frac nm),
    \]
    
    \noindent так как суммы, стоящие в обеих частях равенства, отличаются только порядком следования слагаемых.
    Затем в правую часть последнего равенства вместо $f(m)$ подставим правую часть равенства~\eqref{eq.6}.
    Меняя в получившейся двойной сумме порядок суммирования и применяя лемму~e\eqref{lem.1}, получим следующую цепочку равенств:
    
    \[
        \begin{split}
            \sum_{m|n} \mu(m) f(\frac nm) &= \sum_{m|n} \mu(m) \sum_{k|\frac nm} h(k) =\\
                                          &= \sum_{m|n} \,\,\,\, \sum_{k|\frac nk} \mu(m) h(k) = \sum_{km|n} \mu(m) h(k) =\\
                                          &= \sum_{k|n} \,\,\,\, \sum_{m|\frac nk} \mu(m) h(k) = \sum_{k|n} h(k) \sum_{m|\frac nk} \mu(m) = h(n).
        \end{split}
    \]
    
    Таким образом, справедливость равенства~\eqref{eq.7} установлена.
    Обратное утверждение доказывается аналогично.
    Лемма доказана.
    
    \newtheorem{Th}{Теоркма}
       \begin{Th}
           Для числа $P(n)$ неприводимых многочленов степени $n$ справедливо равенство
           
               \[
                   P(n) = \frac 1n \sum_{m|n} \mu(\frac nm) p^m.
               \]
           
       \label{th.1}
    \end{Th}
    
    {\LargeДоказательство}.
    Из леммы~\eqref{lem.1} следует, что равенство~\eqref{eq.6} леммы~\eqref{lem.3} справедливо при $f(n) = p^n$ и  $h(n) = nP(n)$ для всех натуральных $n$.
    Поэтому утверждение теоремы следует непосредственно из леммы~\eqref{lem.3}.
    Теорема доказана.

\end{document}
